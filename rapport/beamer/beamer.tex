\documentclass[blue]{beamer}
\usetheme{Warsaw}
\setbeamertemplate{navigation
symbols}{\insertframenumber/\inserttotalframenumber}
\usepackage[francais]{babel}
\usepackage[utf8]{inputenc}
\usepackage[T1]{fontenc}
\usepackage{color}
\usepackage{listings}
\usepackage{geometry}
\usepackage{graphicx}

\definecolor{rltgrey}{rgb}{0.9,0.9,0.9}
\definecolor{rltblue}{rgb}{0.5,0.7,0.9}
%\addtobeamertemplate{footline}{\insertframenumber/\inserttotalframenumber}
\lstset{
language=SQL,                   % choose the language of the code
basicstyle=\ttfamily,           % the size of the fonts that are used for the code
numbers=left,              % where to put the line-numbers numberstyle=\footnotesize,      % the size of the fonts that are used for the line-numbers
stepnumber=0,                   % the step between two line-numbers. If it's1each line will be numbered 
numbersep=5pt,                  % how far the line-numbers are from the code
backgroundcolor=\color{rltblue},% choose the background color.
showspaces=false,               % show spaces adding particular underscores 
showstringspaces=false,         % underline spaces within strings
showtabs=false,                 % show tabs within strings adding particular underscores
frame=single,                   % adds a frame around the code
tabsize=1,                      % sets default tabsize to 2 spaces
captionpos=r,                   % sets the caption-position to bottom
breaklines=true,                % sets automatic line breaking
breakatwhitespace=false,        % sets if automatic breaks should only happen at whitespace
title=\lstname,                 % show the filename of files included with \lstinputlisting;also try caption instead of title
escapeinside={\%*}{*)},         % if you want to add a comment within your code
morekeywords={AND,ASC,avg,CHECK,COMMIT,SAVEPOINT,count,DECODE,BEGIN,DESC,DISTINCT,GROUP,IN,LIKE,NUMBER,ROLLBACK,SUBSTR,sum,VARCHAR2}        
}

\title[Système de fichiers distribué]{Système de fichiers distribué : comparaison de GlusterFS, MooseFS et Ceph avec déploiement sur la grille de calcul Grid’5000.}
\author{\it JF. Garcia, F. Lévigne,\\M. Douheret, V. Claudel}
\date{\today}
\begin{document}

%%%%%%%%%%%%%%%%%%%%%%%%%%%%%%%%%%%%%%%%%%%%%%%%%%%%%%%%%%%%%%%%%%%%%%%%%%%%%%%%%%%%%%%%%%%%%%%%
%\begin{frame}
\titlepage
%\end{frame}
%%%%%%%%%%%%%%%%%%%%%%%%%%%%%%%%%%%%%%%%%%%%%%%%%%%%%%%%%%%%%%%%%%%%%%%%%%%%%%%%%%%%%%%%%%%%%%%%
\begin{frame}{Table des Matières}
\begin{columns}[t]
\begin{column}{5cm}
\tableofcontents[sections={1-4}, hideothersubsections]
\end{column}
\begin{column}{5cm}
\tableofcontents[sections={5-8},hideothersubsections]
\end{column}
\end{columns}
%\tableofcontents[hideothersubsection]
%\tableofcontents[subsectionstyle=hide]
\end{frame}
%%%%%%%%%%%%%%%%%%%%%%%%%%%%%%%%%%%%%%%%%%%%%%%%%%%%%%%%%%%%%%%%%%%%%%%%%%%%%%%%%%%%%%%%%%%%%%%%
%%%%%%%%%%%%%%%%%%%%%%%%%%%%%%%%%%%%%%%%%%%%%%%%%%%%%%%%%%%%%%%%%%%%%%%%%%%%%%%%%%%%%%%%%%%%%%%%
%%%%%%%%%%%%%%%%%%%%%%%%%%%%%%%%%%%%%%%%%%%%%%%%%%%%%%%%%%%%%%%%%%%%%%%%%%%%%%%%%%%%%%%%%%%%%%%%
\section{Introduction}
	\subsection{Présentation du sujet}
	\begin{frame}
		\frametitle{Présentation du sujet}
		Comparaison de systèmes de fichiers distribué :
		\begin{itemize}
			\item Système de fichier (FS) : façon de stocker, organiser des informations dans un fichier
			\item Système de fichiers distribué :
			\begin{itemize}
				\item Éclaté sur plusieurs serveurs % éclaté ? Un meilleur mot ?
				\item Disponible depuis plusieurs clients
			\end{itemize}
		\end{itemize}
	\end{frame}

	\subsection{Le Grid'5000}
	\begin{frame}
		\frametitle{Le Grid'5000}
		\begin{itemize}
			\item Infrastructure distribué dédié à la recherche
			\item 11 sites, dont 9 en France
			\begin{figure}
				\includegraphics[width=0.3\linewidth]{../images/Site_map.png}
				\caption{Les sites français du Grid'5000}
			\end{figure}
		\end{itemize}
	\end{frame}

\end{frame}

\section{NFS}
\begin{frame}

\end{frame}

\section{GlusterFS}
\begin{frame}

\end{frame}

\section{MooseFS}
\begin{frame}

\end{frame}

\section{Ceph}
\begin{frame}

\end{frame}

\section{Comparaison}
\begin{frame}

\end{frame}

\section{Conclusion}
\begin{frame}

\end{frame}

\section{Organisation du travail}
\begin{frame}

\end{frame}

\end{document}
